\starttext

Hoi \naam
\blank
Tof dat je een inspanningstest hebt gedaan. Met deze test weten we nu precies jouw comfortzone!
En dat is mooi want met deze zone weet je wanneer je jezelf moet prikkelen om fitter te worden of juist remmen wanneer je te hard werkt.
\blank
Op de volgende pagina's vindt je jouw comfortzone terug plus een heleboel andere statistieken die je niet direct nodig zult hebben maar wellicht leuk vindt om te weten. 
Deze test is er op gericht om je een idee te geven waar je staat. Bijvoorbeeld als je voor een mooie uitdaging wilt trainen en wilt zien of je progressie boekt. 
Of mischien is jouw doel om af te vallen. Met deze gegevens weet je op welke hartslag je dit het beste kan doen.

\pagebreak


\subject{inspanningsgegevens}

Hieronder staan jouw inspanningsgegevens. Staar je vooral niet blind op al deze veelal engelstalige gegevens. Het belangrijkste is dat je een gevoel creeerd zodat je ook zonder metingen effectief jouw ambities kunt halen.

\subsubject{functional threshold power}
Om trainingeffect te meten hebben we een referentiepunt nodig. Dit referentiepunt noemen we functional threshold power, kortweg ftp. De ftp geeft aan hoeveel vermogen je maximaal kunt leveren binnen een uur. Het vermogen is de verhouding tussen de geleverde kracht en trapfrequentie.

Jouw ftp is: \waarde{\bf \ftp W}


\subsubject{power zone}

Jouw intensiteit van een inspanning hangt af in welke power zone je rijdt. Power is de Engelse term voor vermogen, eigenlijk zou een comfortzone een betere benaming zijn. Hieronder zijn jouw zones te zien.
\blank
\bTABLE
\row PinkSherbert 1:\zone{60}W:
\row VividTangerine 2:\zone{67}W:
\row Bittersweet 3:\zone{81}W:
\row PermanentGeraniumLake 4:\zone{90}W:
\row MaximumRed 5:\zone{96}W:
\eTABLE
\blank
Zone 1 gebruik je om te herstellen. Bijvoorbeeld na een inspanningstest. Pas bij zone 3 ga je vermoeidheid merken. Zone 5 kun je beperkt volhouden.

\subsubject{W/kg}
Om jezelf te vergelijken met andere kun je naar de verhouding van je FTP en je gewicht kijken. 
Een profwielrenners zit bijvoorbeeld rond de 6 W/kg, een recreanten zal gemiddeld 1.8 W/kg trappen.
tot 3,5 W/kg zal voor de meeste fanatieke fietsers haalbaar zijn als ze gericht trainen.

Jouw waarde is: \waarde{\bf {\ctxlua{context(math.floor(\ftp/\gewicht*100)/100)}} W/kg}

\subsubject{sweet spot}
Om vooruitgang te boeken moet je niet te weinig inspanning leveren maar ook zeker niet te veel om overbelasting te voorkomen. De sweet spot is precies de juiste ftp waarde om maximaal trainingseffect te behalen bij een beperkte trainingsomvang (dwz. minder dan 3x per week).

Jouw sweet spot is: \waarde{\bf \zone{90} W}

\subsubject{HFmax}
Je hartslag is een goede graadmeter voor jouw huidige toestand. Ben je moe, gestressed of zelf ziek dan zal je dit terug zien aan je hartslag. De maximale hartslag (HFmax) kan als referentiepunt gebruikt worden.

Jouw HFmax is: \waarde{\bf \hfmax bpm}

\subsubject{hartslag zone}

Net als de power zone geeft de hartslag zone jou een indicatie hoe comfortabel je sport. Let er op dat je hartslag niet altijd even betrouwbare weergaven is. dieet of stress kunnen je zones sterk doen veranderen

\blank
\bTABLE
\row PinkSherbert 1:\zoneHF{65} bpm:
\row VividTangerine 2:\zoneHF{70} bpm:
\row Bittersweet 3:\zoneHF{78} bpm:
\row PermanentGeraniumLake 4:\zoneHF{84} bpm:
\row MaximumRed 5:\zoneHF{90} bpm:
\eTABLE
\blank



\subsubject{Vervolg}
Wil je zien of je progressie hebt geboekt dan kun je de test het beste na +- 8 weken nog eens over doen. Mocht je meer gevoel hebben gekregen om op het juiste tempo te rijden dan zou je ook eens kunnen kijken naar een critical power test.
In de Critecal power test maken we een power profiel zodat we weten waar jouw sterken en zwakken kanten zitten. Ook berekenen we je W'. Dit is een waarde welke aangeeft hoe ver je nog boven jouw threshold kan rijden. Bijvoorbeeld handig voor mountainbikers die korte intensieve inspanningen moeten leveren.

Ben je specifiek voor een beklimming aan het trainen dan kunnen ook een klim test doen om een referentiepunt te hebben. 

\stoptext
