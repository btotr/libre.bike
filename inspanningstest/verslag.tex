\title{inspanningstest}


\def\ftp{300}
\def\naam{naam}



\def\zone#1%
  {\ctxlua{context(#1/100*\ftp)}}

\def\row#1 #2:#3:{\bTR[topframe=on,framecolor=#1]
\bTD#2\eTD\bTD#3\eTD
\eTR}

\setupTABLE[frame=on,style=\tfxx]
\setupTABLE[column][1,2][width=0.4 \textwidth]

\starttext
\subject{uitslag}

Hoi \naam

Tof dat je een inspanningstest hebt gedaan. Met deze test weten we nu precies jouw comfortzone!

En dat is mooi want met deze zone weten we wanneer we je moeten prikkelen om fitter te worden of juist remmen wanneer je te hard werkt.

De volgende comfortzones geven aan hoe relaxt je fiets bij een bepaalde
inspanning.

\bTABLE
\row white zone 1:\zone{60}:
\row red zone 2:\zone{75}:
\row white zone 3:\zone{100}:
\row red zone 4:\zone{110}:
\row red zone 5:\zone{120}:
\eTABLE

De basistraining is er op gericht om je fitter te maken. Na 8 weken kun je zone
3 een uur vol houden.

\subject{stats voor nerds}

\subsubject{ftp}
\ftp
\subsubject{cp}
\ftp
\subsubject{pp}
\subsubject{w/kg}
\subsubject{maxHr}
\stoptext
